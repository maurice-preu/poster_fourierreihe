\documentclass[final,t]{beamer}

% poster template
\usepackage[orientation=portrait,size=a4,scale=1.4,debug]{beamerposter}
\usepackage{amsmath, amssymb}
\usepackage[utf8]{inputenc} 
\usepackage[ngerman]{babel} 
\usetheme{zurichposter}
\usepackage[makeroom]{cancel}
\usepackage{xcolor}
\usepackage{siunitx}
\definecolor{MyRed}{rgb}{0.7,0,0}
\renewcommand\CancelColor{\color{MyRed}}
\usefonttheme{serif}

% references
\usepackage[bibstyle=authoryear, citestyle=authoryear-comp,%
hyperref=auto]{biblatex}
\bibliography{references}

% document properties
\title{\LARGE Fourierreihe}
\author{Maurice Preußner, 05.11.2016}


%------------------------------------------------------------------------------
\begin{document}

\begin{frame}{}
\begin{columns}[t]


%-----------------------------------------------------------------------------
%                                                                     COLUMN 1
% ----------------------------------------------------------------------------
\begin{column}{.525\linewidth}

    % Fourierreihe
    \begin{exampleblock}{Fourierreihe}
        Fourierreihe einer periodischen Funktion, die Periode entspricht $T$, es gilt $\omega_0 = \frac{2 \pi}{T}$
        \begin{flalign}
        	f(t) = \frac{a_0}{2} + \sum_{k=1}^{\infty}  (a_k \cdot \cos \Big( \underbrace{ \frac{2 \pi}{T} \cdot k}_{\omega_0 \cdot k} \cdot t \Big) + b_k \cdot \sin \Big( \frac{2 \pi}{T} \cdot k \cdot t \Big) 
        \end{flalign}
	\end{exampleblock}
	
	\begin{exampleblock}{Reelle Fourierkoeffizienten}
		Auf den Faktor $\frac{2}{T}$ wird unten eingegangen
		\begin{align}
			\boldsymbol{a_k} = \frac{2}{T} \int_{0}^{T} f(t) \cdot \cos \Big( \frac{2 \pi}{T} \cdot k \cdot t \Big) \ dt \\
			\boldsymbol{b_k} = \frac{2}{T} \int_{0}^{T} f(t) \cdot \sin \Big( \frac{2 \pi}{T} \cdot k \cdot t \Big) \ dt
		\end{align}
	\end{exampleblock}
	
	\begin{block}{Fourierkoeffizienten für gerade/ungerade Funktionen}
		Für gerade Funktionen mit der Periode T gilt:
		\begin{align}
			\boldsymbol{a_k} = \frac{4}{T} \int_{0}^{\frac{T}{2}} f(t) \cdot \cos \Big( \frac{2 \pi}{T} \cdot k \cdot t \Big) \qquad \qquad \boldsymbol{b_k} = 0  
		\end{align}
		Für ungerade Funktionen mit der Periode T gilt:
		\begin{align}
			\boldsymbol{a_k} = 0 \qquad \qquad \boldsymbol{b_k} = \frac{4}{T} \int_{0}^{\frac{T}{2}} f(t) \cdot \sin \Big( \frac{2 \pi}{T} \cdot k \cdot t \Big) 
		\end{align}
	\end{block}
	
	\begin{exampleblock}{Warum ist $a_k$ für gerade Funktionen null?}
		Die Berechnung der Grundschwingung $b_1$ einer geraden Sinusfunktion $f(t) = \sin(2\pi t)$ sieht folgendermaßen aus:
		\begin{flalign}
			a_k = \frac{2}{T} \int_{0}^{T} \sin(2\pi t) \cdot \cos \Big( \frac{2 \pi}{T} \cdot k \cdot t \Big) \ dt
		\end{flalign}
		Für die Periode und den Faktor $k$ (vielfaches der Grundfrequenz) gilt: $T = 1$ und $k = 1$. Die Integration wird mithilfe der partiellen Integration vorgenommen: $\int u^\prime v \ dx = [uv] - \int u v^\prime $
		\begin{flalign}
			\frac{2}{T} \int_{0}^{T} \sin(2\pi t) \cdot \cos \Big( \frac{2 \pi}{T} \cdot k \cdot t \Big) \ dt =& \int_{0}^{1} \underbrace{\sin(2 \pi t)}_{u^\prime} \cdot \underbrace{\cos(2 \pi t)}_{v} \ dt \\ 	
			\int_{0}^{1} \sin(2 \pi t) \cdot \cos(2 \pi t) \ dt =& \ \Big[ \underbrace{- \frac{1}{2\pi} \cos(2\pi t)}_{u} \cdot \underbrace{\cos(2 \pi t)}_{v} \Big]_0^1 \\ \nonumber
			- \int_{0}^{1} & \underbrace{\cancel{-\frac{1}{2\pi}} \cos(2\pi t)}_{u} \cdot \underbrace{( \cancel{- 2 \pi} \sin(2\pi t))}_{v^\prime} \ dx \\
			\underbrace{2 \int_{0}^{1} \sin(2 \pi t) \cdot \cos(2 \pi t) \ dt}_{2 a_k} =& \ \cdot \Big[- \frac{1}{2 \pi} \cdot \cos^2 (2\pi t) \Big]_0^1 \\
			a_k =& \ - \frac{1}{4\pi} \cdot \Big[\cos^2 (2\pi t) \Big]_0^1 = 0
		\end{flalign}
		Dieser Zusammenhang gilt für alle Frequenzanteile $\omega = k \cdot \omega_0 $, $b_k$ ist für gerade Funktionen stets null.
	\end{exampleblock}

\end{column}


%-----------------------------------------------------------------------------
%                                                                     COLUMN 2
% ----------------------------------------------------------------------------
\begin{column}{.525\linewidth}

    % komplexe Fourierreihe
    \begin{exampleblock}{komplexe Fourierreihe}
    	Fourierreihe einer periodischen Funktion, die Periode entspricht $T$, es gilt $\omega_0 = \frac{2 \pi}{T}$
    	\begin{flalign}
	    	f(t) = \sum_{k=-\infty}^{\infty} c_k \cdot e^{j \cdot \frac{2 \pi}{T} \cdot k \cdot t}
    	\end{flalign}
    \end{exampleblock}
    
    \begin{exampleblock}{Komplexer Fourierkoeffizient}
    	\begin{flalign}
    		\boldsymbol{c_k} = \sum_{k=-\infty}^{\infty} c_k \cdot e^{j \cdot \frac{2 \pi}{T} \cdot k \cdot t}
    	\end{flalign}
    \end{exampleblock}
  
    % Conclusions
    \begin{alertblock}{Conclusions}
        \begin{itemize}
            \item Joint formulation of LP and QP relaxation $\rightarrow$ LPQP.
            \item LPQP solved by a message-passing
            algorithm for modified unary potentials.
            \item Get a smooth objective for free. Key to fast convergence.
            \item Competitive results in terms of MAP state found.
        \end{itemize}
    \end{alertblock}
    
    \begin{exampleblock}{Warum der Faktor $\frac{2}{T}$?}
		Der Faktor $\frac{2}{T}$ ist notwendig um für die Einzelschwingungen betragsmäßig korrekte Amplituden $a_k$ und $b_k$ zu erhalten. 
		\vspace{1ex}
		
		Um diesen Zusammenhang zu erfassen wird von der Funktion $f(t)  = \sin(2\pi t)$ ausgegangen. Die Fourierreihe dieser Funktion besteht aus der Grundschwingung mit einer Frequenz von $f_0 = 1\, \si{Hz}$ und einer Amplitude von $a_1 = 1$.
		\begin{flalign}
		a_1 = 1 = \frac{2}{T} \int_{0}^{T} \sin(2\pi t) \cdot \sin \Big( \frac{2 \pi}{T} \cdot 1 \cdot t \Big) \ dt
		\end{flalign}
		...
    \end{exampleblock}




%    % References
%    \begin{block}{References}
%        \vskip -0.8cm
%        \footnotesize
%        \begin{itemize}
%            \item \fullcite{Pletscher2012}
%        \end{itemize}
%        \normalsize
%        \vskip -0.8cm
%    \end{block}


\end{column}

\end{columns}

\end{frame}

\end{document}
